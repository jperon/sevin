\couplet{
	Tu rayonnais, d'or sur argent,

	Monseigneur de Bouillon jugeant

	Que, pour l'honneur du divin Maître,
	
	Le champ d'argent et la croix d'or
	
	Dans leur splendeur n'étaient encor
	
	Que pauvretés à méconnaître.
	
	Et sur les tours de la cité,
	
	À tous vents, dans l'immensité,
	
	Palpitait la royale enseigne
	
	Qui faisait dire aux musulmans
	
	Que, partout où règnent les Francs,
	
	Jésus-Christ règne.
}


\couplet{
	Malgré l'attrait de la blancheur,
	
	Nous avons choisi pour couleur
	
	Celle du blé qui sort de terre,
	
	Symbole clair, grave leçon,
	
	Pour nous qui sommes la moisson
	
	En laquelle la France espère.
	
	Et sur ce champ d'un vert si doux,
	
	La croix sanglante étend sur nous
	
	Les bras rouges de ses potences :
	
	À s'immoler, et sans regrets,
	
	Ils doivent être toujours prêts,
	
	Les Scouts de France !
}
