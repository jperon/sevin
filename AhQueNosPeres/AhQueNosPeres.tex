\couplet{
	Ah ! que nos pères étaient heureux, \emph{(bis)}
	Quand ils étaient à table !
	Le vin coulait à côté d’eux : \emph{(bis)}
	Ça leur était fort agréable.
}

\refrain{
	Et ils buvaient à pleins tonneaux,
	Comme des trous, \emph{(bis)}
	Morbleu ! bien autrement que nous ! \emph{(bis)}
}

\couplet{
	Ils n’avaient ni riches buffets,
	Ni verres de Venise,
	Mais ils avaient des gobelets
	Aussi grands que leur barbe grise.
}

\couplet{
	Ils ne savaient ni le latin
	Ni la théologie,
	Mais ils avaient le goût du vin,
	C’était là leur philosophie.
}

\couplet{
	Quand ils avaient quelque chagrin
	Ou quelque maladie
	Ils plantaient là le médecin,
	Apothicaire et pharmacie.
}

\couplet{
	Celui qui planta le provins
	Au doux pays de France, 
	Dans l’éclat de rubis du vin
	Il a planté notre espérance.
}
