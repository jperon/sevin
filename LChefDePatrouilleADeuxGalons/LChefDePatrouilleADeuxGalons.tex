\couplet{
	Il est toujours de bonne humeur,
	
	Sauf s'il se fâche, par malheur ;
	
	Il expliqu' tout vraiment très bien,
	
	Excepté quand on n'comprend rien !
}

\couplet{
	De ses dix doigts il sait tout faire :
	
	Un' vis avec un fil de fer,
	
	Et d'un' demi-douzain' d'andouilles
	
	Il fait la plus chic des patrouilles !
}

\couplet{
	C'est un garçon vraiment sérieux,
	
	Jamais sur rien ne ferm' les yeux :
	
	Tous les tours qu'on peut inventer,
	
	Avant nous il les a tentés.
}

\couplet{
	Quand il commence une inspection,
	
	Nous en tremblons tous d'émotion.
	
	Car lorsqu'il nous dit nos défauts,
	
	Y a pas d'vas'line entre ses mots !
}

\couplet{
	D'ailleurs pour tous il est pareil,
	
	Il sait donner un bon conseil,
	
	Ça vous entre tout en blaguant,
	
	Ainsi que la main dans un gant.
}

\couplet{
	Si les petits ont marché trop,
	
	Il les transporte sur son dos ;
	
	Il les dorlote au campement :
	
	C'est un grand frère, une maman.
}

\couplet{
	Autour du feu, le soir, au camp,
	
	Il devient vraiment éloquent :
	
	Il nous raconte des histoires…
	
	Qu'on n'est pas obligé de croire !…
}

\couplet{
	Premier levé, dernier couché ;
	
	Mais, quand le sommeil l'a touché,
	
	Il ronfle auprès d'ses compagnons
	
	Comm' toute une escadrill' d'avions !
}

\couplet{
	Quand il remplace l'Assistant,
	
	Personne n'en est mécontent :
	
	Il remplac'rait le Scoutmaster,
	
	Et le Chef Scout, s'il fallait l'faire !
}

\couplet{
	S'il continue sur ce train-là,
	
	Pour sûr qu'on le canonis'ra,
	
	Et ses Scouts diront à genoux :
	
	« Chef de Patrouill', priez pour nous !… »
}
