\couplet{
	\mbox{Le Totem, dit Chamarande, \emph{\footnotesize bis}}
	
	C'est la dernière invention,...
	
	Dans l'mond' de l'éducation,...
}

\couplet{
	Il doit être, dit l'Oracle,
	
	Un animal du canton
	
	Propre à notre imitation.
}

\couplet{
	Pour pétrir le caractère,
	
	Qu'il ait valeur de symbole,
	
	Et renferme une leçon.
}

\couplet{
	Après ces pro-lé-go-mènes,
	
	Dignes de gens moins frivol's,
	
	Dépêchons et choisissons.
}

\couplet{
	-- J'en veux un qui soit sonore
	
	Éclatant comme un clairon,
	
	\mbox{-- Pourquoi pas comme un klaxon ?}%
}

\couplet{
	\mbox{L'éléphant, ça n'est pas scoute,}
	
	Car ça trompe en tout' saison,
	
	Car ça trompe en tout' saison.
}

\couplet{
	Les lions, c'est trop féroce,
	
	Je dout' que nous en trouvions
	
	Dans les bois des environs.
}

\couplet{
	\mbox{Les renards sont pleins d'astuce,}
	
	Mais ils aiment un peu trop
	
	Certains autres animaux.
}

\couplet{
	Les coucous ? un cri facile,
	
	Mais ils sont trop sans façon
	
	Pour nous voler not' maison.
}

\couplet{
	Le hibou, c'est l'oiseau sage,
	
	Mais il a peur du plein jour ;
	
	C'est le jour que nous cherchons.
}

\couplet{
	L'hirondelle est toute grâce,
	
	Au ciel lance sa chanson,
	
	-- Voilà un' très bonn' raison.
}

\couplet{
	Le conseil de Patrouill' pense
	
	Entendu tout's ces raisons,
	
	Qu'il faut prendre un' décision.
}

\couplet{
	Il suggère par prudence
	
	D'ajourner la solution
	
	À la prochain' réunion.
}

\parbox{\linewidth}{\vspace{3\baselineskip}}
