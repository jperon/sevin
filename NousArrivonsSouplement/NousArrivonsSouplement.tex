\couplet{
	Nous sommes disciplinés\\
	Jusqu'au bout des ongles,\\
	Observant en Loups bien nés\\
	La Loi de la Jungle.\\
	Lorsque \textsc{Raksha} nous emmène,\\
	(Hou ! viv' la Cheftaine !)\\
	Nul ne voudrait l'ennuyer.\\
	(Hou ! viv' le Louv'tier !)\footnotemark
}
\footnotetext{\emph{Variante :} Hou ! viv' le Sizenier !}

\couplet{
	Instruits par le vieux \textsc{Baloo},\\
	Que veut-on qu'il dise,\\
	S'il échappe aux Petits Loups\\
	Quelque… balourdise ?\\
	Devant nos minces fredaines (…)\\
	De notre âge il a pitié. (…)
}

\couplet{
	Pour devenir de bons Loups,\\
	Des bêtes utiles,\\
	\textsc{Bagheera} nous apprend tout,\\
	Tant elle est subtile.\\
	Elle a des jeux par centaines,\\
	Des histoires par milliers.
}

\couplet{
	Quand la lune à l'horizon\\
	Sort du bois qui bouge,\\
	Nous dansons sur le gazon\\
	Devant la Fleur-Rouge\footnotemark :\\
	Les passants qui se promènent\\
	S'arrêtent pétrifiés.
}
\footnotetext{Le feu, en langage de Jungle.}

\couplet{
	Nous imitons de \textsc{Kaa}\\
	Le corps qui se vautre,\\
	Ondulant cahin-caha\\
	L'un derrière l'autre.\\
	Nous sifflons à perdre haleine\\
	Tant que son corps reste entier.
}

\couplet{
	Les flatteurs n'ont pas acquis\\
	Droit à nos suffrages,\\
	Nous chassons les \textsc{Tabaquis}\\
	De notre entourage ;\\
	Des chacals à face humaine\\
	Déblayons notre sentier.
}

\couplet{
	Dieu nous garde d'être amis\\
	Des gens \textsc{Bandar-Lo…gue}\\
	Quel malheur qu'il les ait mis\\
	Dans son catalogue !\\
	Ils courent la prétantaine\\
	Et ne sav'nt pas travailler.
}

\couplet{
	Nous détestons \textsc{Shere Khan}\footnotemark,\\
	Le tigre féroce,\\
	Qui n'a rien d'un gentleman,\\
	Et fait peur aux gosses.\\
	Oui, va, Mowgli nous entraîne,\\
	Nous aurons ta peau rayée !
}
\footnotetext{On prononce \emph{chèr' kân'.}}

\couplet{
	Pour avoir trop aimé l'or\\
	Qui tourne la tête,\\
	Quand Nabuchodonosor\\
	Fut changé en bête,\\
	Sept ans et sept quarantaines,\\
	Il marcha sur quatre pieds.
}

\couplet{
	Mais pour nous c'est différent,\\
	Et Loups que nous sommes,\\
	En jouant on nous apprend\\
	À devenir hommes !\\
	Et cela vaut bien la peine\\
	D'être Louv'teaux quatre années !
}
