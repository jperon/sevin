\couplet{
	Bertrand du Guesclin était preux chevalier :\\*
	Gai ! Du Guesclin !\\
	Il peut aux Routiers enseigner leur métier,\\*
	Il passa par semblable chemin,\\*
	Par semblable chemin.\\
	Gai ! Du Guesclin ! Sut sans nul quartier\\*
	Se donner tout entier,\\*
	Et passa par semblable chemin.
}

\couplet{
	Bertrand du Guesclin était la fleur d'honneur,\\*
	Gai ! Du Guesclin !\\
	Par heur ou malheur, loyal à son seigneur,\\*
	Chevaucha toujours au droit chemin,\\*
	Toujours au droit chemin.\\
	Gai ! Du Guesclin ! Terrible au menteur,\\*
	Oncques ne connut peur,\\*
	Chevaucha toujours au droit chemin.
}

\couplet{
	Bertrand du Guesclin était gai compagnon,\\*
	Gai ! Du Guesclin !\\
	Jouait aux Godons maints tours de sa façon,\\*
	Mais au pauvre ouvrait toujours la main,\\*
	Ouvrait toujours la main.\\
	Gai ! Du Guesclin ! Protecteur des gueux,\\*
	De tous les malheureux,\\*
	Comme toi, au pauvre ouvrons la main.
}

\couplet{
	Bertrand du Guesclin mourut en combattant,\\*
	Gai ! Du Guesclin !\\
	Tous joyeusement souhaitons-nous en autant,\\*
	De mourir les armes à la main,\\*
	Les armes à la main.\\
	Gai ! Du Guesclin ! qui pour Dieu luttiez,\\*
	Apprenez aux Routiers\\*
	À mourir les armes à la main.
}
