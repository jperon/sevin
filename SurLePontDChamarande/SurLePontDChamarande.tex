{\footnotesize

{\centering \textsc{Mouvements}\par}

À chaque couplet, les Petits Loups expriment par gestes un article de leur Loi ou des Maximes, dans l'ordre suivant :

\renewcommand{\couplet}[3][]{%
\item \emph{#1}\emph{Comme ça} : #2 ;\\
\emph{Encor comme ça} : #3.
}

\begin{enumerate}
\couplet{salut de la main droite}{salut des deux mains}
\couplet{on s'accroupit pour le Grand Hurlement}{on saute en l'air, 2\ieme\ temps du Hurlement}
\couplet[Le Petit Loup écoute le Vieux Loup.\\]{les bras le long du corps, fixe, le nez en l'air, le regard tendu vers le ciel}{même attitude du visage, mais \emph{au repos}, jambes écartées, mains derrière le dos}
\couplet[Le Petit Loup ne s'écoute pas lui-même.\\]{les Petits Loups se font face deux à deux, comme pour la Danse de Baloo ; l'un tient l'index levé, l'autre baissant la tête comme s'il recevait une semonce}{mêmes attitudes, en intervertissant les rôles}
\couplet[Le Petit Loup pense d'abord aux autres.\\]{geste de prendre l'argent dans la poche, et de faire l'aumône au voisin de droite}{même geste à gauche}
\couplet[Le Petit Loup ouvre les yeux et les oreilles]{les deux mains formant jumelles devant les deux yeux, en se tournant vers la droite}{les deux mains derrière les deux oreilles, bras écartés, en se tournant vers la gauche}
\couplet[Le Petit Loup est toujours propre.\\]{geste de se laver les dents, avec l'index droit en guise de brosse}{geste de se frotter les genoux l'un près de l'autre, en sautant sur place}
\couplet[Le Petit Loup dit toujours vrai.\\]{la figure aussi rayonnante que possible, le regard droit, une main sur la poitrine, l'autre étendue comme pour un serment, tourné vers la droite}{même geste, tourné vers la gauche, en intervertissant la position des mains.}
\couplet[Le Petit Loup est toujours gai.\\]{bras en l'air, sauter et pirouetter sur place, en tournant vers la droite}{mêmes gestes, vers la gauche}
\end{enumerate}

On termine en faisant le même geste qu'en 1 : saluts, ou en faisant le tunnel du métropolitain : les Petits Loups se mettent deux par deux en se donnant la main ; le premier couple s'écarte en formant arcade avec les bras ; le deuxième passe sous l'arcade en chantant : \emph{Sous le Pont d'Chamarande…} et forme l'arcade devant le 1\ier\ couple, et ainsi de suite jusqu'à ce que tous les couples aient passé.

}
