\couplet{
	Le chef de troupe a dit à ses \textsc{Coucous} :\\
	« Mes bijoux,\\
	Fermez vos becs quand on dit : Garde à vous ! »\footnotemark
}
\footnotetext{On ajoute une unité à la conclusion du refrain après chaque couplet : \emph{Pour 2 jours et 2 nuits,…}}

\couplet{
	Le chef de troupe a dit aux noirs \textsc{Corbeaux} :\\
	« Mes marmots,\\
	Ne mettez pas d'sacs si lourds sur vos dos. »
}

\couplet{
	Le chef de troupe a dit aux \textsc{Hirondelles} :\\
	« Mesd'moiselles,\\
	Pour tirer l'char apportez donc vos ailes ! »
}

\couplet{
	Le chef de troupe a dit aux petits \textsc{Chats} :\\
	« Angoras,\\
	Rentrez vos griff's et marquez bien le pas. »
}

\couplet{
	Le chef de troupe a dit à nos beaux \textsc{Cerfs} :\\
	« Votre flair\\
	Pour bien camper trouvera des clairières. »
}

\couplet{
	Le chef de troupe a dit à nos vieux \textsc{Coqs} :\\
	« Maîtres-Coqs,\\
	Pour le dîner fait's des œufs à la... coque. »
}

\couplet{
	Le chef de troupe a dit aux fins \textsc{Renards} :\\
	« Mes gaillards,\\
	Au dévouement n'ayez pas de retard ! »
}

\couplet{
	Le chef de troupe a dit à nos bons \textsc{Chiens} :\\
	« Veillez bien :\\
	Du camp la nuit vous serez les gardiens ! »
}

\couplet{
	Et tout' la troupe a dit au "scoutmaster" :\\
	« Trève à c't air :\\
	Tous nos gosiers ne demand'nt plus qu'à s'taire !!! »
}
