\couplet{
	\bis{%
	Trois jeunes tambours\\*
	S’en revenaient de guerre,
	}\\
	\emph{Et ri et ran, rapataplan,}\\*
	\emph{S’en revenaient de guerre.}
}

\couplet{
	Le plus jeune a\\*
	Dans sa bouche une rose.
}

\couplet{
	La fille du roi\\*
	Était à sa fenêtre.
}

\couplet{
	Joli tambour,\\*
	Donnez-moi votre rose.
}

\couplet{
	Fille du roi,\\*
	Donnez-moi votre cœur.\footnotemark
}
\footnotetext{Variante : Je te la donne, Mais tu seras ma mie.}

\couplet{
	Joli tambour,\\*
	Demandez à mon père.
}

\couplet{
	Sire le roi,\\*
	Donnez-moi votre fille.
}

\couplet{
	Joli tambour,\\*
	Quelles sont tes richesses ?
}

\couplet{
	Sire le roi,\\*
	Ma caisse et mes baguettes.
}

\couplet{
	Joli tambour,\\*
	Tu n’es pas assez riche.
}

\couplet{
	N’a pas vaillant\\*
	La robe de ma fille.
}

\couplet{
	Sire le roi,\\*
	Je ne suis que trop riche.
}

\couplet{
	J’ai trois vaisseaux\\*
	Dessus la mer jolie.
}

\couplet{
	L’un chargé d’or,\\*
	L’autre de pierreries.
}

\couplet{
	Et le troisième\\*
	Pour promener ma mie.
}

\couplet{
	Joli tambour,\\*
	Dis-moi quel est ton père.
}

\couplet{
	Sire le roi,\\*
	C’est le roi d’Angleterre.
}

\couplet{
	Joli tambour,\\*
	Tu auras donc ma fille.
}

\couplet{
	Sire le roi,\\*
	Je vous en remercie.
}

\couplet{
	Dans mon pays,\\*
	Y en a de plus jolies.
}
