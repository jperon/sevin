\couplet{
  Quand je reçus de l'ordre la cape immaculée
  
  Marquée de la croix rouge, à l'épaule brodée,
  
  Le grand maître, céans, a daigné me parler
  
  \og sois fidèle et ardent car tu es Templier. \fg
}


\couplet{
  Depuis sur terre et mer nous avons guerroyé :
  
  Partout dans le désert, sous le ciel mordoré,
  
  Des sarrasins maudits je me suis fait connaître
  
  Comme un vrai chevalier seul mérite de l'être.
}


\couplet{
  Combien de missions menées jusqu'à leur terme,
  
  Combien d'engagements qui l'ennemi consternent ;
  
  Par le fer de la lance au baucéant sacré,
  
  De Syrie en Provence, j'ai servi Chrétienté !
}


\couplet{
  Or aujourd'hui enfin me voici allongé
  
  Dans de la paille fraîche où j'entends psalmodier ;
  
  Là- haut, dans la chapelle, c'est l'office des morts,
  
  Courage, Dieu t'appelle, tu arrives au port.
}


\couplet{
  O lointaine Champagne, pays de mes aïeux,
  
  Ton ciel ennuagé m'a bien manqué un peu
  
  Sous le firmament bleu et le ciel étoilé
  
  Qu'on voit toute l'année au Crack des chevaliers.
}


\couplet{
  Sur mon honneur, Seigneur, j'ai votre foi jurée,
  
  Je Vous rends mon c\oe ur pur et mon épée sans tâche ;
  
  J'ai combattu pour vous sans repos ni relâche,
  
  Je Vous rends mon épée avec son baudrier.
}


\couplet{
  Sire Dieu protégez ce pays qui est vôtre,
  
  Vous y marchiez jadis suivi de vos apôtres ;
  
  J'ai parcouru ses routes et suivi ses sentiers,
  
  J'ai chevauché sans doute où vous posiez le pied.
}


\couplet{
  La route qui s'achève mène au paradis ;
  
  Saints et Saintes de Dieu, aidez moi en ce jour,
  
  St Georges et St Maurice, qu'il ne soit jamais dit
  
  Que vous m'avez laissé privé du Dieu d'amour.
}


\couplet{
  Sire Dieu de merci, Sire Dieu de bonté :
  
  Dans mon c\oe ur pour un autre il n'y eut jamais place,
  
  Grâce, ô Agneau de Dieu qui toute faute efface
  
  Grâce, Dame Marie à qui l'Ordre est voué.
}
