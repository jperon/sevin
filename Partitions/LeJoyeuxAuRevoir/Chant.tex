\titre[table=Joyeux au-revoir (le)]{Le joyeux au-revoir\footnotemark}\label{AuRevoir}

\footnotetext{L'au-revoir scout n'est pas une cérémonie funèbre : éviter de le rendre trop émouvant. Le rythme est joyeux et rapide, plus encore au refrain.

Former le cercle, croiser les mains, chacun saisissant de sa main gauche la main droite de son voisin. Aucun mouvement des bras pendant le couplet ; scander le \emph{refrain seul} par un mouvement des bras \emph{de haut en bas}, les bras retombant sur la première note de chaque mesure.}

\includely[staffsize=10]{LeJoyeuxAuRevoir.ly}

\needspace{4\baselineskip}

\paroles{LeJoyeuxAuRevoir}
