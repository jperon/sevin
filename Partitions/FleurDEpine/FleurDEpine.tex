\couplet{%
	\bis{%
		Ma mère qui m’a nourrie\\*
		N’a jamais connu mon nom :
	}\\*
	\emph{On m’appelle} \vbis[texte=ter]\ Fleur d’épine,\\*
	Fleur de rose, c’est mon nom !
}

\refrain{
	Tra, la, la, la ! la, la ! la, la…
}

\couplet{
	Fleur d’épine, fleur de rose\\*
	C’est un nom qui coûte cher !\\
	\emph{Car il coûte} la valeur\\*
	De cent écus que j’ai perdus.
}

\couplet{
	Qu’est-ce que c’est que cent écus\\*
	Quand on a l’honneur perdu ?\\
	\emph{Car l’honneur} est privilège\\*
	Des fillettes de quinze ans. 
}

\couplet{
	Ne fais donc pas tant la fière :\\*
	On t’a vue, hier au soir !\\
	\emph{On t’a vue} hier au soir,\\*
	Un beau bourgeois auprès de toi.
}

\couplet{
	Ce n’était pas un bourgeois\\*
	Qui était auprès de moi,\\
	\emph{C’était l’ombre} de la lune\\*
	Qui rôdait autour de moi.
}

\couplet{
	La morale de cette histoire,\\*
	On la sut dix ans plus tard :\\*
	\emph{On la sut} dix ans plus tard,\\*
	Quand elle épousa son bourgeois.
}

\couplet{
	La morale de la morale\\*
	On la sut vingt ans plus tard :\\*
	\emph{On la sut} vingt ans plus tard,\\*
	Quand elle enterra son bourgeois !
}
