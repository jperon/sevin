\newcounter{facteur}
\setcounter{facteur}{11}
%%%%%%%%%% Commandes fournies par ce document : %%%%%%%%%%

%Pour l'inclusion de partitions :
%\partition{Dossier}{Partition}{Mode}

%Pour une rubrique :
%\rubrique{Rubrique}

%Pour un titre (en italique, au fer à droite) : \titre{Titre}

%Pour la traduction d'une partition, en petit au-dessus de la portée : \traduction{Traduction}

%Pour saisir un texte sur une colonne et sa traduction sur la colonne voisine : \traduire{Texte}{Traduction}

%Pour saisir divers symboles :
%Petite croix : \+
%Symbole "verset" : \Vbar
%Symbole "répons" : \Rbar



%%%%%%%%%% Début des hostilités (pour ceux qui veulent modifier les commandes) %%%%%%%%%%

%%Paramètres d'exigence
%\widowpenalty=9999
%\clubpenalty=9999
%\tolerance=7500
%\pretolerance=500

%%Paquetages indispensables

%Partitions de musique
\usepackage{gregoriotex}
\usepackage{gregoriosyms}

%Facteur des dimensions
\setgrefactor{\value{facteur}}

%Images et couleurs
\usepackage{graphicx}
\usepackage{color}
\definecolor{red}{cmyk}{.05,1,1,.1}
\newcommand{\red}{}
\newcommand{\rouge}{%
\renewcommand{\red}%
{\color{red}}}


%Bidouille pour savoir quel est le numéro de ligne
\usepackage[pagewise]{lineno}
\makeatletter
\@addtoreset{linenumber}{page}
\makeatother
\renewcommand\linenumberfont{\color{white}}

%Polices
\newlength{\tailletemp}
\newlength{\taillepolice}
\setlength{\tailletemp}{\grefactor pt}
\setlength{\taillepolice}{1\tailletemp}
\def\policetitre#1{{\fontseries{b}\fontsize{1.2\taillepolice}{1.44\taillepolice}\selectfont #1}}
\newcommand{\entete}[1]{{\fontsize{.7\taillepolice}{.84\taillepolice}\selectfont #1}}
\def\mode#1{{\fontsize{.6\tailletemp}{.8\tailletemp}\selectfont #1}}
\font\libertine=LinLibertine_RB.otf at 2\taillepolice
\newfontfamily\diurnale[Scale=1]{Liturgie}
\newfontfamily\diurnaleB[Scale=1.1]{Liturgie}
\newfontfamily\diurnaleC[Scale=.9]{Liturgie}
\newfontfamily\lettrines[Scale=2.6]{l800}
%\font\diurnale=Liturgie.ttf at \taillepolice
%\font\diurnaleB=Liturgie.ttf at 1.1\taillepolice
%\font\diurnaleC=Liturgie.ttf at .9\taillepolice
%\font\lettrines=l800.ttf at 1.3\tailletemp

%Caractères spéciaux
\catcode`\œ=\active
\def œ{\oe}
%\catcode`\ǽ=\active
%\def ǽ{\'æ}


%Divers
\usepackage{longtable}
\usepackage{parcolumns}


%%Définition de divers paramètres

%En-tête
\fancyhead[RO,LE]{\entete\thepage}

%Espace entre les paragraphes
\setlength{\parskip}{2pt plus 1pt minus 2pt}
%\addtolength{\baselineskip}{0pt plus 5pt minus 5pt}

%Espace entre les colonnes
\newlength{\intercol}
\setlength{\intercol}{.5cm}

%Propriétés des partitions grégoriennes
\newcommand{\dimensionsnormales}{%
\grechangedim{\grespacelinestext}{1.2\taillepolice plus .2\taillepolice}
\grechangedim{\grespaceabovelines}{.7\taillepolice plus .2\taillepolice minus .1\taillepolice}}
\dimensionsnormales
\gresetstafflinefactor{15}
%\grechangedim{\grespacebeneathtext}{1pt}
\def\gretextformat#1{{%\fontseries{b}%
\fontsize{\taillepolice}{\taillepolice}%
\selectfont #1}}
\def\greinitialformat#1{{\lettrines #1}}
\def\grebiginitialformat#1{{%
\raisebox{0pt}[.7\height][0pt]{\lettrinesB #1}}}


%%Commandes personnalisées

%Pour afficher le jour dans l'en-tête :
%\jour{Jour liturgique}
\newcommand{\jour}[1]{%
\newpage
{\centering \policetitre{#1}\par}
\fancyhead[RE,LO]{\entete{#1}}%
\thispagestyle{empty}%
}

%Pour l'inclusion de partitions :
%\partition{Dossier}{Partition}{Mode}
\newcommand{\partition}[3]{%
%\sloppy
\widowpenalty=0
\clubpenalty=0
\smallskip
\gresetfirstlineaboveinitial{\raisebox{.2\taillepolice}{\parbox{3\taillepolice}{\centering\textsc{\textbf{\mode{#3}}}}}}%
{\textsc{\textbf{\mode{#3}}}}
{\includegabcscore{Partitions/Gregorien/#1/#2.gabc}}\par
%\fussy
}

%%Pour un titre (en italique) :
%%\titre{Titre}
%\newcommand{\titre}[1]{
%\vspace{\stretch{1}}\pagebreak[0]
%{\scriptsize\textit{#1}}\\%
%\nopagebreak%\smallskip\nopagebreak%
%}

%Pour un commentaire (gras italique) :
\newcommand{\commentaire}[1]{%
\smallskip
\textit{\textbf{#1}}\par
\nopagebreak\smallskip\nopagebreak
}

%Pour la traduction d'une partition, en petit au-dessus de la portée :
%\traduction{Traduction}
\newcommand{\traduction}[1]{%
\noindent%\parbox{\textwidth}%
{\small #1}\par\nopagebreak}

%Pour saisir un texte sur une colonne et sa traduction sur la colonne voisine :
%\traduire{Texte}{Traduction}
\newcommand{\traduire}[2]{%
\begin{parcolumns}[distance=\intercol,colwidths={1=.487\linewidth},nofirstindent]{2}
\colchunk{\sloppy%
#1\par}
\colchunk{\sloppy%
\small #2\par}
\end{parcolumns}\par\fussy
%\selectlanguage{francais}
%\begin{Parallel}{.47\textwidth}{.48\textwidth}
%\ParallelLText{\noindent#1\par}
%\ParallelRText{\noindent\small #2\par}
%\end{Parallel}
}

%Pour saisir une rubrique :
%\rubrique{Rubrique}
\newcommand{\rubrique}[1]{%
{\noindent\footnotesize{\red #1}\par\nopagebreak%\smallskip\nopagebreak%
}}

%%%%Pour les syllabes d'accent et de préparation dans les Psaumes
%accent
\newcommand{\ac}[1]{\textbf{#1}}
%syllabe de préparation
\newcommand{\pr}[1]{\textit{#1}}
%survenante anticipée d'accent
\newcommand{\sac}[1]{\ac{#1}}
%syllabe de fin, atone
\newcommand{\point}[1]{\textit{#1}}
%dernier accent, ses syllabes de préparation et survenante
\newcommand{\aca}[1]{%
\ifthenelse%
{\equal{\commandkey{ton}}{8G}
\OR \equal{\commandkey{ton}}{8c}
\OR \equal{\commandkey{ton}}{communion-8}
\OR \equal{\commandkey{ton}}{1D}
\OR \equal{\commandkey{ton}}{1f}
\OR \equal{\commandkey{ton}}{1a2}
\OR \equal{\commandkey{ton}}{3a}
\OR \equal{\commandkey{ton}}{5a}
\OR \equal{\commandkey{ton}}{7a}
\OR \equal{\commandkey{ton}}{7c}
\OR \equal{\commandkey{ton}}{Requiem}}%
{\ac{#1}}%
{#1}%
}
\newcommand{\praa}[1]{%
\ifthenelse%
{\equal{\commandkey{ton}}{8G}
\OR \equal{\commandkey{ton}}{8c}
\OR \equal{\commandkey{ton}}{communion-8}
\OR \equal{\commandkey{ton}}{1D}
\OR \equal{\commandkey{ton}}{1f}
\OR \equal{\commandkey{ton}}{3a}
\OR \equal{\commandkey{ton}}{1a2}
\OR \equal{\commandkey{ton}}{Requiem}}%
{\pr{#1}}%
{#1}%
}
\newcommand{\prab}[1]{%
\ifthenelse%
{\equal{\commandkey{ton}}{8G}
\OR \equal{\commandkey{ton}}{8c}
\OR \equal{\commandkey{ton}}{communion-8}
\OR \equal{\commandkey{ton}}{1D}
\OR \equal{\commandkey{ton}}{1f}
\OR \equal{\commandkey{ton}}{1a2}}%
{\pr{#1}}%
{#1}%
}
\newcommand{\prac}[1]{%
\ifthenelse%
{\equal{\commandkey{ton}}{communion-8}}%
{\pr{#1}}%
{#1}%
}
\newcommand{\saca}[1]{#1}
%avant-dernier accent, sa syllabe de préparation
\newcommand{\acb}[1]{%
\ifthenelse%
{\equal{\commandkey{ton}}{5a}
\OR \equal{\commandkey{ton}}{7a}
\OR \equal{\commandkey{ton}}{7c}}%
{\ac{#1}}%
{#1}%
}
\newcommand{\prba}[1]{#1}
%dernier accent du 1° membre, ses syllabes de préparation et survenante
\newcommand{\acc}[1]{%
\ifthenelse%
{\equal{\commandkey{ton}}{8G} \OR \equal{\commandkey{ton}}{8c} \OR \equal{\commandkey{ton}}{communion-8}
\OR \equal{\commandkey{ton}}{communion-1}
\OR \equal{\commandkey{ton}}{1a2}
\OR \equal{\commandkey{ton}}{1D}
\OR \equal{\commandkey{ton}}{1f}
\OR \equal{\commandkey{ton}}{3a}
\OR \equal{\commandkey{ton}}{5a}
\OR \equal{\commandkey{ton}}{7a}
\OR \equal{\commandkey{ton}}{7c}
\OR \equal{\commandkey{ton}}{Requiem}}%
{\ac{#1}}%
{#1}%
}
\newcommand{\prca}[1]{%
\ifthenelse%
{\equal{\commandkey{ton}}{communion-8}}%
{\pr{#1}}%
{#1}%
}
\newcommand{\prcb}[1]{%
\ifthenelse%
{\equal{\commandkey{ton}}{communion-8}}%
{\pr{#1}}%
{#1}%
}
\newcommand{\prcc}[1]{%
\ifthenelse%
{\equal{\commandkey{ton}}{communion-8}}%
{\pr{#1}}%
{#1}%
}
\newcommand{\sacc}[1]{%
\ifthenelse%
{\equal{\commandkey{ton}}{3a}}%
{\sac{#1}}%
{#1}%
}
%avant-dernier accent du 1° membre, sa syllabe de préparation
\newcommand{\acd}[1]{%
\ifthenelse%
{\equal{\commandkey{ton}}{1a2}
\OR \equal{\commandkey{ton}}{communion-1}
\OR \equal{\commandkey{ton}}{1D}
\OR \equal{\commandkey{ton}}{1f}
\OR \equal{\commandkey{ton}}{3a}
\OR \equal{\commandkey{ton}}{7a}
\OR \equal{\commandkey{ton}}{7c}}%
{\ac{#1}}%
{#1}%
}
\newcommand{\prda}[1]{#1}
%un accent spécial pour les versets notés
\newcommand{\aci}[1]{\fontsize{1.1\taillepolice}{1.1\taillepolice}%
%\fontseries{b}%
\selectfont\textbf{#1}}
%syllabes de fin
\newcommand{\pointi}[1]{%
\ifthenelse%
{\equal{\commandkey{ton}}{communion-1}}%
{\point{#1}}%
{#1}%
}
\newcommand{\pointii}[1]{%
\ifthenelse%
{\equal{\commandkey{ton}}{communion-1}}%
{\point{#1}}%
{#1}%
}
\newcommand{\pointiii}[1]{%
\ifthenelse%
{\equal{\commandkey{ton}}{communion-1}}%
{\point{#1}}%
{#1}%
}
\newcommand{\pointiv}[1]{%
\ifthenelse%
{\equal{\commandkey{ton}}{communion-1}}%
{\point{#1}}%
{#1}%
}
\newcommand{\pointv}[1]{%
\ifthenelse%
{\equal{\commandkey{ton}}{communion-1}}%
{\point{#1}}%
{#1}%
}

%Définition des modes

%Pour saisir un verset de Psaume
\newcounter{numero}

\setcounter{numero}{2}
\newcounter{verset}
\newcounter{dernierverset}
\newcommand{\verset}[1]{%
\stepcounter{verset}
\ifthenelse{%
\commandkey{premierverset} = \value{verset} \OR \commandkey{premierverset} < \value{verset} \AND \value{dernierverset} > \value{verset}
}{%
%\parbox[t][\height+.3ex]{\textwidth}{%
\noindent\arabic{numero}.\hspace{1ex}#1%
\par%}%
%
\stepcounter{numero}%
}

}
\newcommand{\versetgloria}[1]{%
\stepcounter{verset}
%\parbox[t][\height+.3ex]{\textwidth}{%
\noindent\arabic{numero}.\hspace{1ex}#1%
%}%
\par%\bigskip
\stepcounter{numero}%
}

%Pour appeler un Psaume, un Cantique, le Gloria
\newkeycommand\psaume[ton=1,premierverset=1,dernierverset,numero=\value{numero}][1]{%
	\setcounter{verset}{0}%
	\setcounter{numero}{\commandkey{numero}}%
	\setcounter{dernierverset}{\commandkey{dernierverset} + 1}%
	\input{Psaumes/Ps-#1}%
}
\newkeycommand\cantique[ton=1,premierverset=1,dernierverset,numero=\value{numero}][1]{%
	\setcounter{verset}{0}%
	\setcounter{numero}{\commandkey{numero}}%
	\setcounter{dernierverset}{\commandkey{dernierverset} + 1}%
	\input{Psaumes/#1}%
}
\newkeycommand{\gloria}[ton=1]{%
	\setcounter{verset}{\value{dernierverset} - 1}
	\input{Psaumes/Gloria}
}
\newkeycommand{\gloriacommunion}[ton=communion-8]{%
	\setcounter{verset}{\value{dernierverset} - 1}
	\bigskip
	\input{Psaumes/Gloria-communion}
}

%Pour saisir divers symboles :
%Petite croix : \+
%Symbole "verset" : \Vbar
%Symbole "répons" : \Rbar
\newcommand{\+}%
{\makebox[2ex][c]%{\includegraphics[width=2.2mm]{crux.pdf}}}
{}}
\newcommand{\vbar}{{\diurnaleC{\red\%.}}}
\renewcommand{\Rbar}{{\diurnale{\red\$}}}
\newcommand{\rbar}{{\diurnaleC{\red\$.}}}
\renewcommand{\grestar}{\raisebox{-.2ex}{\red \diurnaleB *}}
\renewcommand{\gredagger}{{\red \diurnale †}}
\renewcommand{\*}{\raisebox{-.2ex}{\red \diurnale *}}

%Dialogues
\newcommand{\dominus}{\Vbar Dóminus vobíscum.\\\Rbar Et cum spíritu tuo.}
\newcommand{\leseigneur}{Le Seigneur soit avec vous.\\\Rbar Et avec votre esprit.}

%Conclusion des oraisons
\newcommand{\peromnia}{per ómnia s\'æcula sæculórum}
\newcommand{\Peromnia}{Per ómnia s\'æcula sæculórum}
\newcommand{\siecles}{dans tous les siècles des siècles}
\newcommand{\Siecles}{Dans tous les siècles des siècles}
\newcommand{\Perdominum}{Per Dóminum nostrum Iesum Christum Fílium tuum}
\newcommand{\Pereundem}{Per eúndem Dóminum nostrum Iesum Christum Fílium tuum}
\newcommand{\perdominum}{per Dóminum nostrum Iesum Christum Fílium tuum}
\newcommand{\pereundem}{per eúndem Dóminum nostrum Iesum Christum Fílium tuum}
\newcommand{\quitecum}{qui tecum vivit et regnat in unitáte Spíritus Sancti Deus}
\newcommand{\quitecume}{qui tecum vivit et regnat in unitáte eiúsdem Spíritus Sancti Deus}
\newcommand{\quivivis}{qui vivis et regnas cum Deo Patre in unitáte Spíritus Sancti Deus}
\newcommand{\Quivivis}{Qui vivis et regnas cum Deo Patre in unitáte Spíritus Sancti Deus}
\newcommand{\Parjesus}{Par Notre Seigneur Jésus-Christ votre Fils}
\newcommand{\Parlememe}{Par le même Jésus-Christ Notre Seigneur, votre Fils}
\newcommand{\parjesus}{par Notre Seigneur Jésus-Christ votre Fils}
\newcommand{\parlememe}{par le même Jésus-Christ Notre Seigneur, votre Fils}
\newcommand{\quietant}{qui étant Dieu vit et règne avec vous en l'unité du Saint-Esprit}
\newcommand{\quietantm}{qui étant Dieu vit et règne avec vous en l'unité du même Saint-Esprit}
\newcommand{\vousqui}{vous qui vivez avec Dieu le Père en l'unité du Saint-Esprit}
\newcommand{\Vousqui}{Vous qui vivez avec Dieu le Père en l'unité du Saint-Esprit}
\newcommand{\ainsi}{\mbox{Ainsi soit-il}}
\newcommand{\introprefacetraduite}{%
\traduire{\dominus}{\leseigneur}
\traduire{%
  \Vbar Sursum corda.}{%
  \Vbar Élevons nos c\oe urs.}
\traduire{%
  \Rbar Habémus ad Dóminum.}{%
  \Rbar Nous les avons vers le Seigneur.}
\traduire{%
  \Vbar Grátias agámus Dómino Deo nostro.}{%
  \Vbar Rendons grâces au Seigneur notre Dieu.}
\traduire{%
  \Rbar Dignum et iustum est.}{%
  \Rbar Cela est juste et raisonnable.}
\smallskip
}
