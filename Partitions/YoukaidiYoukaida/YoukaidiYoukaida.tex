\couplet{%
	\itshape Puis l'appel et la toilette,\\*
	Youkaïdi, youkaïda,\\
	Et bientôt la tribu prête,\\
	Youkaïdi, aïda,\\
	Offre à Dieu le jour nouveau,\\*
	En saluant le drapeau.\footnotemark
}
\footnotetext{Ce couplet ne figure pas dans la chanson originale.}

\couplet{%
	\emph{Ensuite}, rassemblement,\\*
	Youkaïdi, youkaïda,\\
	Sac au doc, et en avant,\\
	Youkaïdi, aïda,\\
	Nous partons avec courage,\\*
	Transportant notre bagage.
}

\couplet{%
	L'éclaireur en voyageant,\\*
	Youkaïdi, youkaïda,\\
	Peut aller mêm' sans argent,\\
	Youkaïdi, aïda.\\
	Toujours joyeux en chemin,\\*
	Qu'importe le lendemain !
}

\couplet{%
	L'honneur est notre noblesse,\\*
	Youkaïdi, youkaïda,\\
	Un bon cœur, notre richesse,\\
	Youkaïdi, aïda.\\
	Tout droit et toujours sans peur,\\*
	Ainsi marche l'éclaireur.
}

\couplet{%
	Qu'il pleuve ou fasse beau temps,\\*
	Youkaïdi, youkaïda,\\
	Nous sommes toujours contents,\\
	Youkaïdi, aïda.\\
	Bon pied, bon œil, bonne humeur\\*
	Est devise d'éclaireur.
}

\couplet{%
	Quand, chantant un gai refrain,\\*
	Youkaïdi, youkaïda,\\
	Nous passons avec entrain,\\
	Youkaïdi, aïda,\\
	Sur le seuil de la chaumière\\*
	Accourt la famille entière.
}

\couplet{%
	Et si la beauté du site,\\*
	Youkaïdi, youkaïda,\\
	À camper là nous invite,\\
	Youkaïdi, aïda,\\
	Dans les fleurs et l'herbe on tend\\*
	La tente en moins d'un instant.
}

\couplet{%
	Nous faisons notre cuisine\\*
	Youkaïdi, youkaïda,\\
	Bientôt la forêt voisine,\\
	Youkaïdi, aïda,\\
	Laisse filtrer dans ses branches\\*
	De nos feux les fumées blanches.
}

\couplet{%
	Quand le soir étend son voile,\\*
	Youkaïdi, youkaïda,\\
	Dans notre maison de toile,\\
	Youkaïdi, aïda,\\
	Un peu las nous pénétrons\\*
	Aux gais accents du clairon.
}

\couplet{%
	La nuit descend sur la plaine,\\*
	Youkaïdi, youkaïda,\\
	Sans troubler sa paix sereine,\\
	Youkaïdi, aïda.\\
	Sentinelle à l'œil dispos,\\*
	Veille sur notre repos.
}
