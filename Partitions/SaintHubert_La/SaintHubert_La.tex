\couplet{%
	Ô Saint Hubert, patron des grandes chasses,\\*
	Toi qu'exaltait la fanfare au galop,\\
	En poursuivant le gibier à la trace,\\
	Tu le forçais sous l'élan des chevaux.\\
	Nous, les derniers descendants de ta race,\\
	Arrache-nous aux plaisirs avilis ;\\
	Remplis nos cœurs de jeunesse et d'audace,\\*
	Dans la forêt fais-nous chasseurs hardis.
}

\couplet{%
	Sauve d'abord, du bocage à l'Ardenne,\\*
	Notre forêt, si chère aux vieux Gaulois,\\
	Pour qu'à ses chants notre jeunesse apprenne\\
	Les fiers secrets gardés par les grands bois.\\
	Fais nos yeux prompts et fais nos lèvres claires,\\
	Pour bien lancer, quand viendra le danger,\\
	Le cri de chasse ou le dur cri de guerre :\\*
	« Sus à la bête et courons la traquer ! »
}

\couplet{%
	Tu vis un jour, au fond du halier sombre\\*
	Où tes limiers se pressaient aux abois,\\
	La croix du Christ que le grand cerf, dans l'ombre,\\
	Couronnait par l'auréole des bois.\\
	Mystique appel qui conquis ta grande âme :\\
	Tu dis aux coures un méprisant adieu.\\
	Montre à nos yeux cette divine flamme\\*
	Et conduis-nous camper sur les hauts lieux.
}

\couplet{%
	Quand le Seigneur, la chasse terminée,\\*
	Appellera notre nom à son tour,\\
	Epargne-nous les tristes mélopées :\\
	Tu sonneras pour nous le \emph{Point du Jour}.\\
	Au grand galop, pour célébrer ta gloire,\\
	Nous bondirons en poussant l'hallali,\\
	Et nous ferons, au fracas des fanfares,\\*
	En ton honneur trembler le paradis.
}
