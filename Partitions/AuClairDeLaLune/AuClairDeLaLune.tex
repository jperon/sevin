\couplet{%
	Au clair de la lune, mon ami Pierrot,\\*
	Prête-moi ta plume pour écrire un mot.\\*
	Ma chandelle est morte, je n’ai plus de feu,\\*
	Ouvre-moi ta porte, pour l’amour de Dieu.
}

\couplet{%
	Au clair de la lune, Pierrot répondit :\\*
	Je n’ai pas de plume, je suis dans mon lit.\\*
	Va chez ma voisine, je crois qu’elle y est,\\*
	Car dans sa cuisine on bat le briquet.
}

\couplet{%
	Au clair de la lune l’aimable Lubin\\*
	Frappe chez la brune ; elle répond soudain :\\*
	Qui frappe de la sorte ? Il dit à son tour :\\*
	Ouvrez votre porte, pour le Dieu d’amour !
}

\couplet{%
	Au clair de la lune, on n’y voit qu’un peu !\\*
	On chercha la plume, on chercha du feu.\\*
	En cherchant d’la sorte je n’sais c’qu’on trouva ;\\*
	Mais je sais qu’la porte sur eux se ferma !
}
