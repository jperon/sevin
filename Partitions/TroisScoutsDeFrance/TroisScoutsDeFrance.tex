\couplet{%
	Ils se moquaient\\
	De la pluie qui les mouille.
}

\couplet{%
	Une dam' leur dit :\\
	« Vous allez prendre rhume ! »
}

\couplet{%
	« N'ayez pas peur :\\
	Nous dormons sur la plume ! »
}

\couplet{%
	La p'tit' fill' dit :\\
	« J'voudrais bien être leur frère ! »
}

\couplet{%
	Un scout reprit :\\
	« Mam'sell', ça n'peut pas s'faire. »
}

\couplet{%
	Et l'papa dit :\\
	« Vous êtes militaires ? »
}

\couplet{%
	« Pardon Monsieur,\\
	Ce n'est pas notre affaire. »
}

\couplet{%
	« Je vous croyais\\
	D'l'armée américaine. »
}

\couplet{%
	« Nous somm's français,\\
	De Strasbourg, Lille et Rennes. »
}

\couplet{%
	« Je comprends mieux :\\
	C'est des sports que vous faites ? »
}

\couplet{%
	« Pardon, Monsieur,\\
	Nous ne somm's pas athlètes. »
}

\couplet{%
	« Vous êtes au moins\\
	Un' société d'touristes ? »
}

\couplet{%
	« Vous n'y êt's point :\\
	Nous ne somm's pas artistes. »
}

\couplet{%
	« Dites-moi donc\\
	Ce que c'est qu'un boi-scoute ? »
}

\couplet{%
	\emph{« C'est un garçon}\\
	\emph{Qui observ' la loi scoute ! »}\footnotemark
}
\footnotetext{Réponse authentique d'un SM anglais à une paysanne ébaubie, sur le passage d'une troupe, en 1913.}

\couplet{%
	« Ça n' m'expliq' pas\\
	Ce que vous fait's sur terre !!! »
}

\couplet{%
	Le vent narquois\\
	Prit l'chapeau du bonhomme.
}

\couplet{%
	Les Scouts, courtois,\\
	Filèr'nt comme un seul homme.
}

\couplet{%
	On r'prit l'chapeau\\
	au bout d'un kilomètre.
}

\couplet{%
	Au proprio\\
	Tous trois vinr'nt le remettre.
}

\couplet{%
	Et l'plus jeun' dit,\\
	Plein d'une grâce exquise :
}

\couplet{%
	« Notre Patrouille,\\
	Monsieur, se spécialise :
}

\couplet{%
	Car les Scouts, c'est\\
	Une espèce d'école,
}

\couplet{%
	Pour rattraper\\
	Les chapeaux qui s'envolent !!! »
}
