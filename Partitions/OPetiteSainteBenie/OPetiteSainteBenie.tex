\couplet{
	Bénis donc toutes nos patrouilles,\footnotemark\\*
	Thérèse de l'Enfant-Jésus,\\
	Qui s'agenouillent\\*
	À tes pieds nus.
}

\footnotetext{La IX\ieme\ Lille, pour qui avait été composé ce cantique, chante ces deux couplets :

Son foulard que le vent agite, − Pour rappeler ton vêtement, − Ô Carmélite, − Est brun et blanc.

Bénis donc la Neuvième Lille, − Thérèse de l'Enfant-Jésus, − Qui t'est docile − On ne peut plus.}

\couplet{
	Conserves-y si bien la grâce\\*
	Que nul de ceux qu'elle a reçus\\
	Jamais ne fasse\\*
	Pleurer Jésus.
}

\couplet{
	Par ta pureté ravissante,\\*
	Que nous portions en Paradis\\
	L'âme innocente\\*
	Des tout-petits.
}

\couplet{
	Nul peintre n'a jamais su rendre\\*
	Ton sourire venu du Ciel ;\\
	Viens nous l'apprendre,\\*
	Fleur du Carmel.
}

\couplet{
	Fais-nous suivre ta simple voie\\*
	Et, toujours souriants et doux,\\
	Semer la joie\\*
	Autour de nous.
}

\couplet{
	Ton regard était si limpide,\\*
	Donne-nous ta sincérité\\
	Et ta candide\\*
	Humilité.
}

\couplet{
	Fais que nous nous aimions en frères,\\*
	Ainsi que les premiers chrétiens\\
	Car, tous nos frères,\\*
	Ils sont les tiens.
}

\couplet{
	Donne-nous une âme si grande\\*
	Que, t'imitant de notre mieux,\\
	Nul ne marchande\\*
	Rien au Bon Dieu !
}

\couplet{
	Toi qui désirais le martyre,\\*
	Donne-nous d'aimer Dieu bien fort\\
	Et de sourire\\*
	À notre mort.
}

\couplet{
	Qu'un jour, Troupe et Meute complète,\\*
	Tous ayant bien gardé la Loi,\\
	Jésus nous mette\\*
	Tout près de toi !
}


