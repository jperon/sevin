\refrain{
	Dans le soir d'or résonne, résonne,\\
	Dans le soir d'or résonne le cor.\\
	Résonne, résonne, résonne le cor. \vbis\\
	Dans le soir d'or résonne, résonne,\\
	Dans le soir d'or résonne le cor.
}

\couplet{%
	C'est le cor du grand Roland,\\
	Qui sonne, affolant,\\
	Sous le ciel sanglant.\\
	C'est le cor du roi Saint Louis\\
	Sonnant l'hallali\\
	Des païens maudits.\\
	C'est le cor du gai Duguesclin\\
	Harcelant sans fin\\
	L'Anglais qui le craint.
}

\couplet{%
	C'est le cor de Jeanne Lorraine\\
	Qui sonne et s'égrène\\
	Dans la nuit sereine.\\
	C'est le cor du preux Bayard\\
	Qui dans le brouillard\\
	Rallie les fuyards.\\
	C'est le cor qui sonne le jour\\
	Où la gloire accourt :\\
	Condé, Luxembourg.
}

\couplet{%
	C'est le cor de Hoche et Marceau,\\
	Des gas en sabots\\
	Sauvant nos drapeaux.\\
	C'est le cor du vieil Empereur\\
	Qui sonne et se meurt\\
	Dans l'île des pleurs.\\
	C'est le cor des chasseurs de fer\\
	Tenant quatre hivers\\
	Des Vosges à l'Yser.
}

\couplet{%
	Et c'est le cor du grand chef Maud'huy,\footnotemark\\
	Dont l'âme aujourd'hui\\
	Toujours nous conduit !
}
\footnotetext{Ce dernier couplet se chante debout, en hommage à la mémoire de notre premier Chef-Scout (̣† 16 juillet 1921).}
