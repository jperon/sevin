\couplet{%
	Le chef avait des ch'veux noirs d'ébène,\\*
	Le s'cond était tout rose et tout blond ;\\
	Au chef des yeux couleur de verveine,\\*
	Au s'cond des yeux couleur de charbon.
}

\couplet{%
	L'chef mesurait un mètr' vingt à peine,\\*
	Le s'cond était un' perche à houblon ;\\
	L'un possédait un' petit' bedaine,\\*
	L'autre n'avait pas trac' de bedon.
}

\couplet{%
	La voix du chef : un souffle, une haleine,\\*
	La voix du s'cond : un vrai coup d'canon ;\\
	Et quand le chef avait le cœur en peine,\\*
	Le s'cond s'tordait comme un tir'bouchon.
}

\couplet{%
	Le chef avait des idées soudaines,\\*
	Le s'cond était plein de circonspection.\\
	Mais entre eux deux, la chose est certaine,\\*
	Régnait toujours la plus douce union.
}

\couplet{%
	Le chef disait : « La Meut'se promène »,\\*
	Le s'cond disait : « Gardons la maison. »\\
	Le chef disait : « Mettez des bas d'laine »,\\*
	Le s'cond disait : « Non, des bas d'coton ! »
}

\couplet{%
	Quand l'chef voulait camper dans la plaine,\\*
	Le s'cond allait camper sur les monts ;\\
	Le chef trouvait la marmit' trop pleine,\\*
	Le s'cond faisait doubler les rations !
}

\couplet{%
	Mais pour tous deux le grave problème\\*
	Était celui de la direction :\\
	Ils divergeaient, c'est là chose humaine,\\*
	Tout en étant d'accord sur le fond !
}

\couplet{%
	Le chef disait : J'voudrais un' cheftaine,\\*
	Les parisiens trouv'nt que c'est très bon…\\
	Le s'cond disait : Prends-en deux douzaines,\\*
	Un' seul' culott' vaut vingt-quatr' jupons !
}

\couplet{%
	Au bout d'un an, de tout' la Sizaine,\\*
	Trois étaient morts, trois à Charenton :\\
	− C'est c'qu'on appell' l'Systèm' des Sizaines,\\*
	− Essayez-le, l'Système a du bon.
}
