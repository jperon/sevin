\couplet{
	Au trente et un du mois d’août, \emph{(bis)}\\*
	Nous aperçûmes sous l’vent à nous \emph{(bis)}\\*
	Une frégate d’Angleterre,\\
	Qui fendait la mer et les flots ;\\*
	C’était pour attaquer Bordeaux.
}

\refrain{
	Buvons un coup, buvons en deux, (bis)\\*
	À la santé des amoureux, (bis)\\*
	À la santé de roi de France ;\\
	Et zut ! pour le roi d’Angleterre,\\*
	Qui nous a déclaré la guerre !
}

\couplet{
	Le capitaine au même instant\\*
	Fit appeler son lieutenant :\\
	« Voilà l’Anglais, t’sens-tu l’courage\\*
	D’aller l’attaquer à son bord\\*
	Savoir qui sera l’plus fort ? »
}

\couplet{
	Le lieutenant, fier et hardi,\\*
	Lui répondit : « Capitaine, oui !\\
	Faites monter tout l’équipage :\\*
	Hardis gabiers, fiers matelots,\\*
	Faites monter tout l'monde en haut. »
}

\couplet{
	Vire lof pour lof en arrivant,\\*
	Nous l’attaquâmes par son avant.\\
	À coups de sabres, a coups de haches,\\*
	De pics, de couteaux, d’mousquetons,\\*
	Nous l’avons mis à la raison.
}

\couplet{
	Que va-t-on dire de lui tantôt,\\*
	En Angleterre et à Bordeaux,\\
	Pour s’être ainsi laissé surprendre\\*
	Par un brigantin d’six canons,\\*
	Lui qu’en comptait trente six et bons ?
}
