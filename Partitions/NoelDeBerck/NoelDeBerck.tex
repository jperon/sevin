\let\vcouplet\couplet
%\renewcommand{\couplet}[2][Le jour de Noël, de grand matin.]{\vcouplet{
%	#2\\*
%	Le jour de Noël \vbis\\
%	#2\\*
%	#1
%}}
\renewcommand{\couplet}[2][Le jour de Noël, de grand matin.]{\vcouplet{
	#2\\*
	#1
}}

\couplet{C'étaient trois barques à voiles d'or}

\couplet[Couverts de brocarts et d'orfrois.]{La première portait trois rois,}

\couplet[Vêtus en très pauvres pèlerins.]{La deuxième, trois prêtres saints,}

\couplet{Et la troisième, trois mamans,}

\couplet[Qu'à Berck on disait descendu.]{Et tous cherchaient l'Enfant-Jésus}

\couplet[Sous les traits d'un petit louveteau.]{L'ont trouvé dans les hôpitaux,}

\couplet[Et semblait cloué sur une croix.]{Dormait sur un cadre de bois,}

\couplet[Et c'est à quoi l'ont tous reconnu.]{Il était pauvre et presque nu,}

\couplet[Et remontèrent dans leur bateau.]{Les Rois lui firent leurs cadeaux,}

\couplet[Puis sont partis bien loin le prêcher.]{Les prêtres saints l'ont approché}

\couplet[L'ont embrassé tout simplement.]{Mais en pleurant, les trois mamans}

\couplet[N'est jamais plus sorti du port.]{Et leur navire aux voiles d'or}

\couplet{J'ai vu trois nefs entrer au port}
