\couplet{%
	C’était un soir, sur les bords de l’Yser,\\*
	Un soldat belge qui montait la faction ;\\
	Vinrent à passer trois gardes militaires,\\
	Parmi lesquelles était le roi Albert.\\
	« Qui vive là ? » lui crie la sentinelle,\\
	« Qui vive là ? vous ne passerez pas !\\
	Si vous passez, craignez ma baïonnette ;\\
	Retirez-vous, vous ne passerez pas ! » \vbis\\*
	\emph{Halte là !}
}

\couplet{%
	Le roi Albert, en fouillant dans sa poche ;\\*
	« Tiens, lui dit-il, et laisse-moi passer.\\
	− Non, non ! lui dit la brave sentinelle,\\
	L’argent n’est pas pour un vrai soldat belge !\\
	Dans mon pays, je cultivais la terre,\\
	Dans mon pays, je gardais les brebis ;\\
	Mais maintenant que je suis militaire,\\*
	Retirez-vous, vous ne passerez pas. »
}

\couplet{%
	Le lendemain, au grand conseil de guerre,\\*
	Le roi Albert lui demanda son nom ;\\
	« Tiens, lui dit-il, voilà la croix de guerre,\\
	La croix de guerre et la décoration.\\
	− Que va-t-elle dire, ma bonne et tendre mère,\\
	Que va-t-elle dire en me voyant si beau :\\
	La croix de guerre est à ma boutonnière\\*
	Pour avoir dit : « Vous ne passerez pas ! »
}
