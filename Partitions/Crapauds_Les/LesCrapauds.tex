\couplet{%
	La nuit est limpide,\\*
	L’étang est sans ride,\\*
	Dans le ciel splendide\\*
	Luit le croissant d’or ;\\
	Orme, chêne ou tremble,\\*
	Nul arbre ne tremble,\\*
	Au loin le bois semble \\*
	Un géant qui dort.\\
	Chien ni loup ne quitte\\*
	Sa niche ou son gîte,\\*
	Aucun bruit n’agite\\*
	La terre au repos ;\\
	Alors dans la vase,\\*
	Ouvrant en extase\\*
	Leurs yeux de topaze\\*
	Chantent les crapauds.
}

\couplet{%
	Ils disent : « Nous sommes\\*
	Haïs par les hommes,\\*
	Nous troublons leurs sommes\\*
	De nos tristes chants ;\\
	Pour nous, point de fêtes,\\*
	Dieu seul, sur nos têtes,\\*
	Sait qu’il nous fit bêtes\\*
	Et non point méchants.\\
	Notre peau terreuse\\*
	Se gonfle et se creuse,\\*
	D’une bave affreuse\\*
	Nos flancs sont lavés ;\\
	Et l’enfant qui passe\\*
	Loin de nous s’efface,\\*
	Et, pâle, nous chasse\\*
	À coups de pavés. »
}

\couplet{%
	« Des saisons entières\\*
	Dans les fondrières,\\*
	Un trou sous les pierres\\*
	Est notre réduit.\\
	Le serpent en boule\\*
	Près de nous s’y roule,\\*
	Quand il pleut, en foule\\*
	Nous sortons la nuit ;\\
	Et dans les salades\\*
	Faisant des gambades,\\*
	Pesants camarades,\\*
	Nous allons manger,\\
	Manger sans grimace,\\*
	Cloporte ou limace,\\*
	Ou ver qu’on ramasse\\*
	Dans le potager. »
}

\couplet{%
	« Nous aimons la mare\\*
	Qu’un reflet chamarre,\\*
	Où dort à l’amarre\\*
	Un canot pourri ;\\
	Dans l’eau qui la mouille,\\*
	Sa chaîne se rouille,\\*
	La verte grenouille\\*
	Y cherche un abri.\\
	Là, la source épanche\\*
	Son écume blanche,\\*
	Un vieux saule penche\\*
	Au milieu des joncs ;\\
	Et les libellules\\*
	Aux ailes de tulle\\*
	Font crever les bulles\\*
	Au nez des goujons. »
}

\couplet{%
	« Quand la lune plaque\\*
	Comme un vernis laque\\*
	Sur la calme flaque\\*
	Des marais blafards,\\
	Alors, symbolique\\*
	Et mélancolique,\\*
	Notre lent cantique\\*
	Sort des nénuphars. »\\
	La nuit est limpide,\\*
	L’étang est sans ride,\\*
	Dans le ciel splendide\\*
	Luit le croissant d’or ;\\
	Orme, chêne ou tremble,\\*
	Nul arbre ne tremble,\\*
	Au loin le bois semble \\*
	Un géant qui dort.
}
