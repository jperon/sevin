\couplet{%
	La maman du petit homme\\*
	Lui dit un matin :\\
	« À 16 ans t'es haut tout comme\\*
	Notre huche à pain.\\
	À la ville tu peux faire\\*
	Un bon apprenti ;\\
	Mais pour labourer la terre,\\*
	T’es ben trop petit, mon ami,\\*
	T’es ben trop petit, dame oui ! »
}

\couplet{%
	Vit un maître d’équipage\\*
	Qui lui rit au nez,\\
	En lui disant : « Point n’engage\\*
	Les tout nouveaux-nés.\\
	Tu n’as pas laide frimousse,\\*
	Mais t’es mal bâti :\\
	Pour faire un tout petit mousse,\\*
	T’es cor' trop petit, mon ami,\\*
	T’es cor' trop petit, dame oui ! »
}

\couplet{%
	Dans son palais de Versailles\\*
	Fut trouver le Roy :\\
	« Je suis gars de Cornouailles,\\*
	Sire, équipez-moi ! »\\
	Mais le bon Louis XVI,\\*
	En riant, lui dit :\\
	« Pour être garde française,\\*
	T’es ben trop petit, mon ami,\\*
	T’es ben trop petit, dame oui ! »
}

\couplet{%
	Cependant la guerre éclate\\*
	Au printemps suivant,\\
	Et Grégoire entre en campagne \\*
	Avec Jean Chouan.\\
	Les balles passaient nombreuses\\*
	Au dessus de lui,\\
	En sifflotant, dédaigneuses :\\*
	« Il est trop petit, ce joli,\\*
	Il est trop petit, dame oui ! »
}

\couplet{%
	Cependant une le frappe\\*
	Entre les deux yeux ;\\
	Par le trou l’âme s’échappe,\\*
	Grégoire est aux Cieux.\\
	Là, saint Pierre, qu’il dérange,\\*
	Lui dit « Hors d’ici !\\
	Il nous faut un grand archange :\\*
	T’es ben trop petit, mon ami\\*
	T’es ben trop petit, dame oui. »
}

\couplet{%
	Mais en apprenant la chose\\*
	Jésus se fâcha,\\
	Entrouvrit son manteau rose\\*
	Pour qu’il s’y cachât ;\\
	Fit entrer ainsi Grégoire \\*
	Dans son Paradis,\\
	En disant : « Mon Ciel de gloire,\\*
	En vérité je vous le dis,\\*
	Est pour les petits, dame oui ! »
}
