\couplet{
	Un jour la troupe campa, \emph{a a a}\\*
	La pluie s’mit à tomber, \emph{b b b}\\*
	L’orage a tout cassé, \emph{c c c}\\*
	Failli nous inonder, \emph{a b c d}.
}

\couplet{
	Le chef s’mit à crier, \emph{é é é}\\*
	À son adjoint Joseph, \emph{f f f}\\*
	Fais-nous vite à manger, \emph{g g g}\\*
	Les scouts sont sous la bâche, \emph{e f g h}.
}

\couplet{
	Les « pinsons » dans leur nid, \emph{i i i}\\*
	Les « loups » dans leur logis, \emph{j j j}\\*
	Chahutèrent, quel fracas ! \emph{k k k}\\*
	Avec les « hirondelles », \emph{i j k l}.
}

\couplet{
	Joseph fit de la crème, \emph{m m m}\\*
	Et du lapin d'garenne, \emph{n n n}\\*
	Et même du cacao, \emph{o o o}\\*
	Mes amis quel souper ! \emph{m n o p}.
}

\couplet{
	Soyez bien convaincus, \emph{q q q}\\*
	Que la vie au grand air, \emph{r r r}\\*
	Fortifie la jeunesse, \emph{s s s}\\*
	Renforce la santé, \emph{q r s t}.
}

\couplet{
	Maintenant qu'y n'pleut plus, \emph{u u u}\\*
	Les scouts vont se sauver, \emph{v v v}\\*
	Le temps est au beau fixe, \emph{x x x}\\*
	Plus besoin qu’on les aide ! u v x z.\footnotemark
}
\footnotetext{En certains endroits, on ajoute, sur l'air du dernier verset :
\emph{Et on n'a rien trouvé, pour "double-v" ; Et puis, comme chez les Grecs, y'a pas d'"i-grec" !}}
