\couplet{
	Amis, il faut faire une pause,\\*
	J’aperçois l’ombre d’un bouchon.\\
	Buvons à l’aimable Fanchon,\\*
	Chantons pour elle quelque chose.
}

\refrain{
	Ah ! Que son entretien est doux,\\*
	Qu’elle a de mérite et de gloire ;\\
	Elle aime à rire, elle aime à boire,\\*
	Elle aime à chanter comme nous,%
	\raisebox{1.5ex}[0pt][0pt]{\hspace{3.7em}\emph{\} \emph{ter}}\hspace{-3.7em}}\\*
	Oui comme nous.
}

\couplet{
	Fanchon, quoique bonne chrétienne,\\*
	Fut baptisée avec du vin,\\
	Un Bourguignon fut son parrain,\\*
	Une Bretonne sa marraine.
}

\couplet{
	Fanchon préfère la grillade\\*
	À d’autres mets plus délicats.\\
	Son teint prend un nouvel éclat\\*
	Quand on lui verse une rasade.
}

\couplet{
	Fanchon ne se montre cruelle\\*
	Que lorsqu’on lui parle d’amour ;\\
	Mais moi, je ne lui fais la cour\\*
	Que pour m’enivrer avec elle.
}
