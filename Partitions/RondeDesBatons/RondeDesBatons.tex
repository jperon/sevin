\couplet{
	Qui vient à notre aide\\
	Et sert de remède\\
	Dans les accidents ?\\
	Qu'un chien nous attaque,\\
	Le bâton-matraque\\
	Lui brise les dents.\\
	Cassez-vous la jambe,\\
	Oh ! les scouts ingambes,\\
	Vite, vos vestons !\\
	Comme une civière\\
	Est facile à faire\\
	Avec nos bâtons !
}

\couplet{
	Un cas de détresse,\\
	Le bâton s'empresse\\
	De le signaler ;\\
	Il trace la piste\\
	Et montre au touriste\\
	Où il doit aller.\\
	Lorsque la patrouille\\
	Dans le bois qu'on fouille\\
	Avance à tâtons,\\
	Par les nuits obscures\\
	Nos marches sont sûres\\
	Avec nos bâtons.
}

\couplet{
	Ils servent de perche\\
	À celui qui cherche\\
	Le gué d'un torrent ;\\
	Prenez-les pour chaise,\\
	Vous serez à l'aise\\
	Deux heures durant.\\
	Lorsqu'il pleut à verse\\
	Et que l'eau nous perce,\\
	Nous nous abritons\\
	En traçant, pratiques,\\
	Un cercle magique\\
	Avec nos bâtons.
}

\couplet{
	Vient l'heure où l'on dîne,\\
	Pour notre cuisine\\
	Formant les faisceaux,\\
	Nous mettons bien vite\\
	Sur le feu marmite,\\
	Gamelles et seaux.\\
	Quand la nuit approche,\\
	À coups de mailloche\\
	Gaîment nous plantons,\\
	Au vent palpitantes,\\
	Nos toiles de tentes,\\
	Avec nos bâtons.
}

\couplet{
	Qui savent les lire\\
	Trouvent à s'instruire\\
	En les inspectant :\\
	Ils marquent notre âge\\
	Et nos sauvetages\\
	Et nos campements,\\
	Car les arabesques\\
	Fines ou grotesques\\
	Que nous y sculptons\\
	C'est tout un grimoire :\\
	Toute notre histoire\\
	Est dans nos bâtons.
}

\couplet{
	Faire leur éloge\\
	Tout un tour d'horloge\\
	Serait incomplet :\\
	Par défaut de place,\\
	Nous vous faisons grâce\\
	Du dernier couplet.\\
	Voyez donc en somme,\\
	Mesdam's, Messieurs, comme\\
	Nous vous respectons :\\
	Toute notre troupe\\
	Vous salue en groupe\\
	Avec ses bâtons.
}
