\refrain{
	Ah ! Les p’tits potes, ah ! Les p’tits potes,\\*
	Ah ! Les p’tits potes, ah ! Les p’tits potes\\*
	Ah ! Les p’tits potes, potes, potes !
}

\couplet{
	C’lui qu’est l’plus grand\\*
	\emph{Chez les p’tits potes,}\\*
	Ils l’appellent tous\\*
	Le grand pot’haut !
}

\couplet{
	C’lui qu’est l’plus bête\\*
	Ils l’appellent tous\\*
	Le p’tit compote !
}

\couplet{
	C’lui qu’est l’plus vieux\\*
	Ils l’appellent tous\\*
	Le pote âgé !
}

\couplet{
	C’lui qui commande\\*
	Ils l’appellent tous\\*
	L’omnipotent !
}

\couplet{
	C’lui qu’est l’plus gras\\*
	Ils l’appellent tous\\*
	Le p’tit pot’lé !
}

\couplet{
	Quand les p’tits potes\\*
	\emph{Vont en Espagne}\\*
	Ils s’écrient tous :\\*
	« Les p’tits pot’ olé ! »
}

\couplet{
	Le plus savant\\*
	Ils l’appellent tous \\*
	Le p’tit potache !
}

\couplet{
	Le plus pansu\\*
	Ils l’appellent tous\\*
	L’hippopot’âme !
}

\couplet{
	S’y a une bêtise\\*
	On peut êt’ sûr\\*
	Que l’compote y est !
}

\couplet{
	C’lui qu’est l’plus riche\\*
	Ils l’appellent tous\\*
	Le pot’pourri !
}

\couplet{
	Quand y’a un repas\\*
	Ils s’y r’trouv’ tous\\*
	Les potes en tas !
}

\couplet{
	C’lui qu’est l’plus triste\\*
	Ils l’appellent tous\\*
	Le p’tit pot’terne !
}

\couplet{
	S’y a un incendie\\*
	Ils z’y courent tous\\*
	Les potes au feu !
}

\couplet{
	S’y a des ennuis\\*
	Il faut qu’ils trouvent\\*
	Le pote aux roses !
}

\couplet{
	Quand y a un pote\\*
	\emph{Qui veut s’marier}\\*
	Il faut qu’il trouve\\*
	La pote et ose !
}

\couplet{
	S’il a une belle fille\\*
	Heureux celui\\*
	Que la pote aime !
}

\couplet{
	Si t’as des sous \\*
	\emph{Ne les prêt’ pas}\\*
	À n’importe qui \\*
	Mais z’au pote ami !
}

\couplet{
	Quand y a un mort\\*
	Ils disent qu’ils ont\\*
	Un pote en ciel !
}

\couplet{
	Si t’es un pote\\*
	\emph{T’iras au ciel}\\*
	Puisqu’on dit qu’tous\\*
	Les pot’iront !
}
