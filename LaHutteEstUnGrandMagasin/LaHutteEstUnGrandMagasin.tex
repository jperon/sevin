\couplet{
	On vend de très larges chapeaux,
	
	Des chemis's larg's comm' des pal'tots,
	
	Et des culottes... dilatées,
	
	Mais la largeur dans les idées,
	
	On n'en vend pas.
}

\couplet{
	On vend des foulards aveuglants,
	
	Et des fanions rouges sanglants,
	
	Mais la simplicité d'allure,
	
	On n'en vend pas.
}

\couplet{
	On vend des bâtons, des bérets,
	
	Des nœuds d'épaule bigarrés,
	
	Couleur de ciel, couleur de rouille ;
	
	Mais le bon esprit de patrouille,
	
	On n'en vend pas.
}

\couplet{
	On vend des courroies de foulard,
	
	Des badg's qui sont des œuvres d'art,
	
	Des cordelières, des aigrettes ;
	
	Mais des Bonnes Actions toutes faites,
	
	On n'en vend pas.
}

\couplet{
	On vend des tent's pour tous les goûts,
	
	Des canadienn's, des marabouts,
	
	Petites, grandes ou moyennes ;
	
	Mais des recett's pour qu'elles tiennent,
	
	On n'en vend pas.
}

\couplet{
	On vend des haches de campeur,
	
	Et des empreintes de traqueur,
	
	Des boussoles, des porte-cartes ;
	
	Mais l'art de n'pas perdre la carte,
	
	On n'en vend pas.
}

\couplet{
	On vend des quarts, on vend des seaux,
	
	Et des marmites Bonnamaux
	
	(La seul', l'uniqu', la véritable !)
	
	Mais l'art de faire un plat mangeable,
	
	On n'en vend pas.
}

\couplet{
	On vend des bouquins excellents
	
	Pour chanter et \emph{penser scout'ment},
	
	Des flûtiaux et des varinettes,
	
	--- Des idées justes, des voix nettes,
	
	On n'en vend pas.
}

\couplet{
	On vend des tas de manuels,
	
	Pansements individuels,
	
	Teinture d'iode en bonbonne ;
	
	Mais l'art de ne tuer personne,
	
	On n'en vend pas.
}

\couplet{
	On vend l'Examen d'Aspirant,
	
	L'art de saluer en douz' temps
	
	(À moins que ce ne soit en seize ?)
	
	Mais la politesse française,
	
	On n'en vend pas.
}

\couplet{
	Avec tout cet équipement,
	
	Avec le nouveau Règlement,
	
	Votre liste est enfin complète,
	
	Mais l'esprit scout point ne s'achète :
	
	On n'en vend pas.
}
