\newcommand{\lpl}[3]{%
\couplet{
	#1#2\\*
	\emph{− #2, #2 −}\\
	#1#2,\\*
	#3
}
}

\lpl{Était toujours }{mal peigné}{Ses habits pas soignés.}

\lpl{Sa casquette é}{tait si sale}{Qu'il n'y avait pas plus sale.}

\lpl{Sous sa casquette }{un museau}{De petit moricaud.}

\lpl{Son grand foulard }{brun et blanc}{Tenait on ne sait comment\footnotemark.}
\footnotetext{Modifier le couplet suivant le besoin.}

\lpl{Portait un af}{freux chandail}{Qui partait maill' par maill'.}

\lpl{Sa culotte é}{tait d'un bleu}{Extrêmement douteux.}

\lpl{Il exhibait }{deux genoux}{Toujours couverts de boue.}

\lpl{Et ses bas à }{revers verts}{Étaient mis de travers.}

\lpl{Pour finir, ses }{godillots}{Par vingt trous prenaient l'eau.}

\lpl{Bref, était si }{mal fic'lé}{Qu'on lui dit de s'en aller.}

\lpl{Ce fut un grand }{désespoir}{Comme vous pouvez croire.}

\lpl{Mais quinz'jours a}{près nous vint}{Un nouveau très très bien.}

\lpl{Qui disait : J'veux }{zêt' louv'teau}{Et je m'appell' Jeannot.}

\lpl{Ses ch'veux étaient }{bien peignés}{Ses habits fort soignés.}

\lpl{Il avait u}{ne casquette}{Tout' neuve et très coquette.}

\lpl{Sous sa casquette }{il montrait}{Un minois fort propret.}

\lpl{Son beau foulard }{brun et blanc}{Était très élégant.}

\lpl{Portait un jo}{li chandail}{Qui n'perdait pas ses maill's.}

\lpl{Sa culott' d'un }{bleu profond}{N'avait pas d'trous au fond.}

\lpl{Ses petits ge}{noux lavés}{Luisaient comm' des pavés.}

\lpl{Et ses bas à }{revers verts}{Avaient leurs deux jarr'tières.}

\lpl{Enfin deux sou}{liers bien noirs}{Lui servaient de miroirs.}

\lpl{On l'reconnut }{malgré ça}{Et la Meut' l'acclama.}

\lpl{Plus tard il fut }{désigné}{Pour devenir Siz'nier.}

\lpl{Au camp pour nous }{z'endormir}{Racontait ses souv'nirs :}

\lpl{Y avait un' fois }{un Louv'teau}{Qui s'appelait Jeannot…}
