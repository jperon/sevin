\couplet{
	Le juif ayant tout pris,\\*
	Tout raflé dans Paris,\\*
	Dit à la France :\\*
	« Tu n’appartiens qu’à nous,\\*
	Obéissance !\\*
	Tout le monde à genoux ! »
}

\refrain{\emph{\footnotesize 1\ier\ refrain :}\\*
	Non, non, la France bouge,\\*
	Elle voit rouge,\\*
	Non, non,\\*
	Assez de trahison !
}

\couplet{
	« Tant pis », dit le rabbin,\\*
	« Je tiens tout dans ma main :\\*
	J’ai la police,\\*
	Et pour violer la loi,\\*
	Une justice,\\*
	De magistrats sans foi. »
}

\couplet{
	Les travailleurs ont faim,\\*
	Le juif dit : « pas de pain,\\*
	Mais à rafales,\\*
	Pour sauver nos écus,\\*
	Voici des balles :\\*
	Peuple ne bouge plus ! »
}

\couplet{
	De brûler nos vaisseaux,\\*
	Avec nos arsenaux,\\*
	Le juif est maître,\\*
	Sous les canons prussiens,\\*
	Dreyfus le traître\\*
	Pousse nos citoyens.
}

\couplet{
	Assez de Panama !\\*
	Assez de Thalamas !\\*
	Toute ta clique\\*
	De pédants, de brigands,\\*
	Ô république,\\*
	Nous la foutrons dedans !
}

\refrain{\emph{\footnotesize 2\ieme\ refrain :}\\*
	Une, deux ! La France bouge,\\*
	Elle voit rouge,\\*
	Une, deux,\\*
	Les français sont chez eux !
}

\couplet{
	Juif insolent, tais-toi,\\*
	Voici venir le Roi,\\*
	Et notre race\\*
	Court au devant de Lui :\\*
	Juif, a ta place,\\*
	Notre roi nous conduit !
}

\couplet{
	Le Roi revient  d’exil :\\*
	« O France, dira-t-il,\\*
	Reine du monde,\\*
	Te voila donc aux mains\\*
	Du juif immonde,\\*
	Coureur de grands chemins ?  \\*
	La France bouge (suite)
}

\couplet{
	Oui, la France aux français,\\*
	A mes loyaux sujets,\\*
	Je tiens le glaive,\\*
	Pour que le travailleur\\*
	En paix achève\\*
	Son honnête labeur. »
}

\couplet{
	Notre jeunesse en fleur\\*
	Vous a donné son cœur,\\*
	Roi magnanime,\\*
	Menez-là jusqu’aux Cieux,\\*
	De cîme en cîme,\\*
	Sur vos pas glorieux.
}

\couplet{
	Hardi ! France d’abord !\\*
	Français, mieux vaut la mort\\*
	Que l’esclavage.\\*
	Gloire à qui tombera !\\*
	Tous à l’ouvrage,\\*
	La France renaîtra !
}

\couplet{
	Demain sur nos tombeaux,\\*
	Les blés seront plus beaux :\\*
	Formons nos lignes !\\*
	Nous aurons, cet été,\\*
	Du vin aux vignes,\\*
	Avec la royauté !\\*
}
